\label{sec:abstract}
\begin{center}
\section*{Abstract}
\end{center}

Shortest path routing schemes can lead to hot-spots or bottlenecks on some links, even when resources remain under-utilized elsewhere in the network. Hence, adapting the routing to network conditions through Traffic Engineering (TE) is essential for the efficient use of network resources. For this purpose, Internet Service Providers (ISPs) currently deploy TE over routing protocols, using techniques such as Multi-Protocol Label Switching (MPLS). However, such a deployment generate additional protocol overhead and have high reaction time to changes in the network traffic. Integrating TE capabilities with the routing scheme can bring about faster reaction to traffic variations as well as better utilization of resources. We developed a decentralized routing protocol based on the idea of backpressure routing, BP STEP-UP,  which adapts to traffic variations in the network. Experimental results show that the modified protocol can improve the performance of the basic backpressure protocol by upto 50\% in link congestion and 32\% in convergence time.



