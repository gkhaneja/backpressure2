\label{sec:related}
\section{Related work}

Tassiulas and Ephremides discovered that queue length is an indicator of congestion in the network while exploring max-weight scheduling in wireless networks and leveraged this information for maximizing throughput in wireless environment~\cite{BP-orig}. The original backpressure algorithm was the extension of max-weight scheduling algorithm in multi-hop wireless networks. This algorithm was independently discovered as a solution to multi-commodity flow problem by Leighton et. al.~\cite{Leighton1}. McKeown et al.~\cite{nick1} extended this idea to input-queued switches and introduced the notion of throughput optimal routing in wired networks. Although the idea of backpressure routing had been introduced more than two decades ago, it is not widely used in practice due to poor delay performance caused by routing loops. In spite of its shortcomings, we propose to explore this direction further since backpressure is the only proven technique that can achieve throughput optimal routing in a distributed manner without a priori knowledge about the traffic patterns. While protocols such as modified OSPF~\cite{mOSPF}, DEFT~\cite{DEFT}, PEFT ~\cite{PEFT} etc. combine traffic engineering with routing, they involve centralized computation which prevents a completely distributed deployment. On the other hand, backpressure algorithm can be implemented as a totally decentralized protocol.

Several modifications have been introduced to improve the performance of the traditional backpressure algorithm. Backpressure Control Protocol~\cite{BCP} used in sensor networks replaces FIFO service with LIFO service to improve delay performance. The importance of using shortest paths with backpressure to reduce delay was first put forward by Neely et al.~\cite{Neely1}. Another work~\cite{SP1} also notes that lack of a metric that indicates closeness to the destination, is a cause for poor delay performance of opportunistic protocols such as backpressure. Multiple variations of backpressure protocol that rely on the notion of shortest paths to reduce delays have been proposed after Neely~\cite{Neely1} introduced the idea. Ying et al.~\cite{Austin1} uses the shortest path information by maintaining additional queues at each node corresponding to hop-counts. Packet-by-packet adaptive routing ~\cite{Srikant3} also combines shortest path and backpressure routing to achieve high performance. It provides an elegant solution for separating routing and scheduling by introducing the notion of shadow packets. Separating the backpressure computation from the real packet routing using shadow packets was first proposed in ~\cite{Srikant1} and ~\cite{Srikant2}. The notion of backpressure has also been explicitly combined with shortest path routing in ~\cite{BP-lcn}.

As an optimization, we intend to use the idea of proportional sharing of links introduced by Walton~\cite{walton} to speed up the convergence of backpressure routing.