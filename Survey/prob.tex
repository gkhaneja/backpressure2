\label{sec:prob}


%\begin{center}
\section*{Problem Statement}
%\end{center}

Routing decisions of shortest path routing protocols are solely based on pre-assigned costs in the network and are agnostic to the real traffic. This can lead to congestion in some parts of the network and under-utilization elsewhere. We can achieve greater responsiveness to traffic variations and freedom from hot-spots by relying on protocols that combine distance information with knowledge about congestion in the network. 

Backpressure routing protocol~\cite{BP-orig} is based on the idea that queue lengths provide direct indication of congestion in the network. Each node in the network decides the next hop for incoming traffic by comparing its queue length with that of its neighbors. The largest difference in queue length, i.e. the largest gradient, will occur along the best path towards destination. Thus, each node forwards traffic based on queue length information received from its immediate neighbors. 

Backpressure routing scheme is throughput optimal,i.e., if the incoming traffic is within the capacity region of the network, the protocol will route it successfully. In spite of its throughput efficiency, backpressure protocol is not widely used in practice due to several limitations. First, the protocol does not consider path lengths. Forwarding decisions are solely based on local congestion information. This causes routing loops and delays in the network. Second, each node has to maintain separate queues for every destination in the network. This is impractical for conventional switches with limited buffer space. However, the protocol does provide interesting features such as throughput-optimal routing and congestion-awareness. 

Recently, several ideas have been proposed ~\cite{Srikant3, Austin1} that combine the notion of shortest path routing with backpressure protocol. This opens an exciting arena for congestion-aware shortest path routing. In particular, one of the variants~\cite{Srikant3} also obviates the need for multiple queues per node. But the efficiency of the algorithm is not flushed out completely due to the limited simulations performed on it. The theoretical foundation also needs further work to be developed into a fully-functional distributed protocol, suitable for practical realization. Our project aims at extending this work to develop a throughput-optimal congestion-aware distributed routing scheme. We will optimize the protocol to speed up convergence and avoid routing delay with minimal communication overhead. We will design and deploy an optimized variant of the basic backpressure protocol in a testbed and analyze its performance with realistic traffic patterns.  
