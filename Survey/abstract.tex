\label{sec:abstract}
\begin{center}
\section*{Abstract}
\end{center}

Shortest path routing schemes can lead to hot-spots or bottlenecks on some links, even when resources remain under-utilized elsewhere in the network. Hence, adapting the routing to network conditions through Traffic Engineering (TE) is essential for the efficient use of network resources. For this purpose, Internet Service Providers (ISPs) currently deploy TE over routing protocols, using techniques such as Multi-Protocol Label Switching (MPLS). However, such a deployment generate additional protocol overhead and have high reaction time to changes in the network traffic. Integrating TE capabilities with the routing scheme can bring about faster reaction to traffic variations as well as better utilization of resources. Our project involves developing and testing a distributed routing protocol capable of handling traffic variations in the network. The proposed protocol, inspired by the idea of backpressure routing, will route traffic along the best path based on conventional routing metrics (path length) as well as  congestion in the network.
